\documentclass[10pt,twoside,BCOR=12mm,DIV=classic]{scrbook}
\usepackage[ngerman]{babel}
\usepackage[T1]{fontenc}
\usepackage[utf8]{inputenc}

\begin{document}
\title{Eine Sammlung von Schriften und Gebeten}
\author{Bruder Maritius von Pratanberg\\Leitender Scriptor im Orden von Schwert und Kelch}
\date{im Jahre des Herren 533}

\maketitle
\tableofcontents

\chapter{Die Geburt des Lichts I}
"Ein Herz das die Liebe fand rief mich, um all jene aus der Finsternis zu führen,
die bereit waren zu sehen"
- Sheenallasaar, in der Ersten Feste, 12 Jahre vor dem Armageddon

Dies ist die Geschichte eines Sterblichen, der in der Dunkelheit geboren wurde
und in Finsternis herrschte. Die Geschichte eines Sterblichen der entdeckte, was
es bedeutet zu sehen. Die Geschichte eines Sterblichen, der bereit war sein
Herz zu öffnen, um Ordnung ins Chaos zu bringen und um die Bedeutung des
Lebens erkennen zu können. Dies ist die Geschichte meines Vaters Nandur, aus
den Ebenen von Aquillon. Und so berichtete er mir von der Zeit seiner
Herrschaft in der Finsternis.
- Aitharos der Sohn des Nandur

"Eine Schlacht mehr ist geschlagen, die Letzten meines eigenen Blutes
vernichtet. Nun bin ich der Herrscher des Seelenthrons Barghaans. Voll Stolz
und Macht sollte mein Herz sein, doch fühle ich mich leer und all meiner Ziele
beraubt. Mit Sicherheit wird bald ein Anderer meinen Platz einnehmen und von
neuem das Land in Krieg und Chaos stürzen so wie es seit Anbeginn der Zeit war,
denn nur der Stärkste unter den Kindern Barghaans kann über sie herrschen!" Dies
waren die Worte die mein Vater einst sprach, bevor er im Licht des Ausmaßes
seiner Fehler gewahr wurde.

"Doch was geschieht mit mir? Voll Stärke und grausamer Erwartung sollte ich mich
darauf freuen, den Nächsten zu empfangen und ihn in tausend Teile zu
zerschmettern, denn nur so wird der Ruhm unseres Schöpfers hinausgetragen in die
Weiten der Welt, um einen jeden anzustacheln sich am ewigen Chaos zu Ehren
unseres Herrn zu beteiligen. Immer schrecklicher und finsterer werden die
Kämpfe! Brüder gegen Schwestern, Söhne gegen Mütter? Allein was für ein Wert hat
das Streben nach dem Thron absoluter Macht, wenn es nie von Dauer ist?
Soll dies alles sein wofür wir leben?
Nein! Nein! Nein!"

Und so begann alles. In diesem Augenblick erkannte Nandur.Und er sagte einst zu
mir es wäre gewesen, als sei sein Gesicht mit kaltem Wasser benetzt worden, doch
an keiner anderen Stelle seines Körpers fühlte er diesen warmen Regen. Dies war
das erste Mal in seinem Leben, dass er erkannte, zu welch vielseitigen Gefühlen
wir Sterblichen fähig sind. Trauer. Trauer, um all jene, welche durch seine Hand
getötet wurden, Trauer um die vergangene Zeit, zu der er noch als kleines Kind
in den Armen seiner Mutter lag und noch nichts wusste von all dem Morden und
Vernichten das ihn Barghaans Glauben lehrte. Trauer, als er auf den erschlagenen
Körper seiner Mutter zu seinen Füßen hinab blickte, die er selbst zuvor vom
Seelenthron gestoßen hatte.
Und mit jedem Herzschlag sehnte er sich mehr danach, zuerfahren, was dies wohl
für ein Gefühl war, als er noch in ihren Armen lag, und von ihr in den Schlaf
gewiegt wurde.

Verzweifelt und unsicher, überrannt von der Flut dieser Gefühle erhob er sich
vom Thron der Finsternis, streckte seine Hände empor und öffnete die Pforten
seines Herzens um die Frage seiner Existenz Barghaans selbst entgegen zu
schleudern: "Gib mir einen Namen für die Gefühle, die tief in mir verborgen
sind! Sag mir, was mein Herz so sehr beengt und begehrt zugleich. Sag mir, warum
meine brennenden Augen meine Haut mit warmem, salzigem Blut benetzen? Warum
Blut? Warum verlangst du all dieses Morden und die ewige Vernichtung von uns?"

Wissend, das diese ketzerischen Worte mit Sicherheit die ewige Verdammnis
bedeuteten, wartete Nandur auf die vernichtende Antwort des Dijai'dan, des
ersten Sohnes Barghaans. Weit riss er seine Augen auf, um die Dunkelheit zu
durchdringen. Um sehen zu können. Und mein Vater wuste nicht woher dieses Wort
karm, noch verstand er sie. Denn er wuste nicht, was es bedeutet zu sehen, noch
wuste er das noch etwas anderes existiert als die Finsternis.

"Auf einmal durchdrangen meinen Kopf Schmerzen ungeahnter Stärke. Laut schrie
ich auf und war bereit zu sterben. Meine Augen schienen sich in meinen Schädel
zu bohren und doch lebte ich noch! Ich war geblendet von einem Gleißen. Licht,
wie es noch nie zuvor in der Finsternis zu sehen gewesen war, schien auf mich
hinab und erfüllte meinen Verstand mit tausenden von Worten. Worten, die noch
keinen Sinn für mich ergaben. Doch als ich bereits glaubte, dass mich dieses
Licht zu verbrennen schien, bildete sich aus den vielen eine klare Stimme und
sprach laut zu mir. Und mit jedem Wort, dass in meinem Geist erschallte,
schienen sich meine Augen Stück für Stück zu öffnen." Dies waren die Worte, mit
denen Nandur später von der Offenbarung des Herren berichtete.

Und der Herr sprach, und für einen Augendblick verstummte die Welt um ihm zu
lauschen: "Nandur, Kind Barghaans, du gabst mir die Möglichkeit, in das Chaos
und die Dunkelheit die Dich und Dein Volk umgibt einzugreifen. Wisse das mein
Name is Bahamuth ist, Sohn Yols und Bruder deines Schöpfers. Seit Äonen
betrachte ich seine Schöpfung ohne Hoffnung in ihr zu sehen. Doch jetzt gabst du
mir die Möglichkeit, Deinem Geschlecht den Weg des Lichts zu eröffnen. Ich
setzte Euch einst den Samen des Lichts in Eure Herzen und wartete darauf, dass
eines Tages ein Sterblicher, nur ein einziger, den Samen in sich finden würde.
Fast schon hatte ich jegliche Hoffnung verloren, doch dann warst du es, der mich
als erster rief. Du bist derjenige, der Licht und Schatten gleichermaßen in sich
trägt. Du, der Grausamste unter den Kinder Barghaans, hast die Finsternis in
Deinem Herzen besiegt und mich gerufen. So werde ich Dir einen Weg zeigen. Einen
Weg aus der Finsternis ins Licht."

Und als das letzte Wort in seinem Kopf verhallt war, öffnete er seine Augen ganz
und sah den Drachen aus Platin, dessen Augen zwei goldenen Sonnen glichen. In
diesem Augenblick erhob sich der Körper meines Vaters wie von selbst, und
Bahamuth sprach erneut zu ihm während er dem Licht entgegenschwebte.

"Nie wieder sollst Du knien vor der Finsternis. Nie wieder sollst Du Dich
verbeugen müssen, vor den Geißeln des Lebens!
Nichts als die fünf großen Tugenden des Lichts sollst Du ehren:
Stärke und Geist,
Frieden und Gerechtigkeit,
Liebe und Großmut,
Ordnung uns Weisheit,
Leben und Wahrheit.
Jene sollst Du ergründen und diejenigen darüber lehren, die zum Lichte wollen,
so wie Du es willst. Genauso ermahne ich Dich, diese Lehren niemals mit Gewalt
zu verbreiten, aber wenn es sein muss dafür zu sterben. Denn selbst wenn Dich
der Feind niederstreckt, schenke ihm die Gnade Deines wissenden Herzens."

Und mit diesen Worten gab Bahamuth eines seiner goldenen Augen um für immerdar
die Welt in Licht zu tauchen, so dass es einem jeden ein Zeichen sein sollte,
ein Symbol der Hoffnung. Von diesem Tag an zog es über den Himmel, geschaffen
aus dem Leibe und Atem Bahamuths, seine Bahnen ziehend, für alle Zeiten verfolgt
und beneidet von der Dunkelheit. So zeigte er seinem dunklen Bruder auf ewig,
was für Grausamkeiten sich in der Dunkelheit verbargen. Und ein drittes und
letztes Mal sprach der Herr des Lichtes:

"Dein Lehrer soll meine Tochter sein. Sheenallasaar, Tochter des Lichts und des
Lebens, Herrin der Kinder der Lüfte und Behüterin der Eleevaar, den Kindern des
Lebens."

Kaum waren die Worte Nandurs wahrgewordenen Verlangens nach Wahrheit und Wissen
verhallt, da sah er eine Gestalt unter den mächtigen Schwingen Bahamuths
hervortreten. Sheenallasaar, Schönheit und Reinheit in ihrer ursprünglichsten
Form. Groß und majestätisch, sanft und zärtlich, stark und zugleich voller
Weichheit war ihre Gestalt. Augen, die ihn voll unverdorbenem Leben und
unendlichem Wissen ansahen, erfüllt von Gefühlen, die mNandur bis zu diesem
Augenblick noch völlig fremd gewesen waren. er glaubte, sein Herz müsse
zerspringen, zerreißen und vergehen, gegenüber dieser unendlichen Fülle von
Empfindungen.

Emporgehoben von all diesen Gefühlen und neuen Erfahrungen blickte Nandur hinab
vom Seelenthron und all das Grauen welches sein Blick erfasste raubte ihm den
Atem. Tausend Fuß hoch stand er auf einem gigantischen Berg, geschaffen aus den
Leibern unzähliger Opfer. Alles Opfer der gewaltigen Schlacht um den Thron der
Finsternis.

Getötet lagen sie danieder. Getötet seine Mutter. Getötet sein Vater. Getötet
all seine Geschwister. Getötet von seiner Hand. Getötet von seinem unendlichen
Streben nach Macht.
Das erste Mal in seinem Leben erblickte er die Gesichter all derer, die er
erschlagen hatte und eine schier unüberwindliche Welle der Trauer und
Verzweiflung überkam ihn. Denn er spürte den Verlust von Liebe. Liebe die er nie
würde zeigen können, die ihm für immer verwehrt sein würde.
Mit Tränen in den Augen stand Nandur auf dem Berg des Wahnsinns und des
unendlichen, alles verachtenden Hasses.

Er wollte mein Schwert ergreifen um sich selbst zu entleiben. Doch in jenem
Augenblick als der Schmerz um all die Verlorenen ihn zu überwältigen drohte,
berührte ihn eine Hand. Und es schien ihm, als würde sie sein Innerstes tröstend
umfassen und seinem Herz die Kraft geben weiter zu schlagen trotz all der Schuld
die auf ihm lastete. Langsam drehte er sich um und wurde gefangen genommen von
den unendlichen Augen Sheenallasaars. Und Nandur sprach: "Lass mich in die
Dunkelheit gehen, wohin ich gehöre. Zurück an den Ort der Buße, bis in alle
Ewigkeiten."

"Nein, Nandur, Goldener, der Du geblickt hast in die Augen Bahamuths. Gehe
nicht! Durch Dich habe ich erfahren, was Trauer ist. Und erst diese Trauer zeigt
mir, wie wertvoll die Liebe ist, was für ein Geschenk in dem Gefühl der Hoffnung
liegt. Du zeigst mir das Besondere in dem für mich Allgegenwärtigen und
Natürlichen. Erst jetzt verstehe ich, wie viel ich von den Sterblichen lernen
kann. Trotze der Dunkelheit die Dein Herz ergriffen hat! Öffne die Tore Deiner
Seele und erschaffe Dein eigenes Licht! Ein Licht aus dem Stärke und Geist,
Frieden und Gerechtigkeit, Liebe und Großmut, Ordnung uns Weisheit, Leben und
Wahrheit hervortreten werden, um Dir und den Deinen den Weg des Lichts zu
weisen. Ein jedes Wesen birgt eine Stärke, egal wie schwach sie auch sein mag,
welche du jedoch nie erreichen kannst."

Und gemeinsam stiegen wsie hinab vom Cullam Toroll, geschaffen aus den Leichen
der Kinder Barghaans, einzig und allein zu Ehren der Finsternis. Dies war der
Augenblick, in dem mein Vater das erste Mal sah, wie das schönste und
wundervollste aller Wesen weinte. So nahm er sie in seine Arme und schöpfte
Kraft aus ihrer Wärme und gab zugleich seine Liebe und Stärke um sie zu schützen
vor dem endlosen Wehklagen der tausenden und abertausenden dahin geschlachteter
Seelen.
Bahamuth erkannte die Pein, welche sie zu ergreifen und hinab zu ziehen drohte,
und so vollbrachte er ein weiteres Wunder, ein drittes Zeichen seiner
unendlichen Barmherzigkeit. Der gesamte Berg erstrahlte im gleißenden Licht der
Reinheit Bahamuths und die Leiber wurden zu Stein und ein Weg formte sich vor
unseren Füßen, so dass sie hinab steigen konnten.

Vieles lernten sie von einander als sie hinab stiegen in die einst finstere Welt
auf der Suche nach Menschen und Geschöpfen, die sich ebenfalls nach dem Geschenk
des Lichts sehnten. Kein Tier versuchte sie anzugreifen, viel zu sehr waren sie
fasziniert von der Gestalt der Sheenallasaar. So kam es, dass die ersten Wesen,
die die Schönheit des Lichts erblickten, die Tiere waren. Denn von nun an,
sollte kein Lebewesen mehr in Finsternis leben.

Und so brachen meine Eltern vom Fuße des Berges auf, um das Licht hinaus in die
Welt zu tragen.
\end{document}
